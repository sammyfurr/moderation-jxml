\documentclass[12pt]{article}
\usepackage{graphicx}
\usepackage{listings}
\usepackage{color}

\author{Sammy Furr}
\title{Moderation Project: JSON to XML using Flex and Bison}
\date{\today}

\graphicspath{ {./images/} }

\definecolor{sgreen}{rgb}{0.52,0.53,0}
\definecolor{gray}{rgb}{0.5,0.5,0.5}
\definecolor{scyan}{rgb}{0.16,0.63,0.60}

\lstset{frame=tb,
  language=C,
  aboveskip=3mm,
  belowskip=3mm,
  showstringspaces=false,
  columns=flexible,
  basicstyle={\small\ttfamily},
  numbers=none,
  numberstyle=\tiny\color{gray},
  keywordstyle=\color{sgreen},
  commentstyle=\color{gray},
  stringstyle=\color{scyan},
  breaklines=true,
  breakatwhitespace=true,
  tabsize=3
}

\begin{document}
\begin{titlepage}
\maketitle
\end{titlepage}
\section{Introduction}
For the fifth lab in Design of Programming Languages, we were asked to
write a program to translate JSON to XML.  To do this, we used a
scanning and parsing system called SLLGEN which was created by one of
the two books for the class, Essentials of Programming
Languages.\cite[pg. 379]{eopl} Written in Scheme, SLLGEN uses macros
to turn Scheme representations of regular expressions and grammars
into lexers and parsers.  Though SLLGEN works, the error reporting
leaves a lot to be desired, and I was interested in writing a lexer
and parser using more industry-standard tools.  I've written a lexer,
parser, and interpreter using Flex and Bison, two open-source and
industry-standard tools.  This paper aims to explore the differences
between writing a scanner and parser in SLLGEN vs. in Flex+Bison.  I
will explore the importance of language and tool choice when writing
metaprograms.
\section{Comparison}
\subsection{Language Differences}
Since JSON is a relatively simple language to lex and parse, the
largest difference between using SLLGEN and Flex+Bison was the
language the \textit{tools} were implemented in.  SLLGEN is written in
Scheme, whereas Flex and Bison are written in C.  This gives a
significant advantage to SLLGEN for those looking to program quickly,
as the time necessary to write a JSON to XML translator after the
lexer and parser are finished will be lower in Scheme than in C.  In
addition, those who don't wish to worry about memory management will
be better served using SLLGEN--even simply lexing JSON in C requires
memory management to handle JSON strings.
\subsection{SLLGEN vs. Flex: Lexing}
The implementation of the lexer in Flex and SLLGEN is relatively
similar---both take a list of regular expressions and their
corresponding tokens.  In Flex, this matches a JSON number:
\begin{lstlisting}
  -?[0-9]*\.?[0-9]+([eE][-+]?[0-9]+)?
  { yylval.NUM = atof (yytext); return NUM; }
\end{lstlisting}
And this is part of the code to do the same using SLLGEN:
\begin{lstlisting}
(jnumber ((or (concat "-" digit)
              digit)
          (arbno digit)
          ... number)
\end{lstlisting}
\par
Though regular expressions in SLLGEN are represented using Scheme
functions, they work fundamentally the same way.  Flex gives a bit
more power when creating tokens--we can decide what, if any, value we
want the token to have (yylval).  The real advantage to using Flex
over SLLGEN came when parsing JSON strings.  The specification for
JSON strings states that a JSON string is a quote, followed by
anything until another quote is reached, with control characters
escaped.\footnote{Appendix A contains the JSON string specification.}
The notion of ``any non-quote character'' turns out to be impossible
to represent using SLLGEN.  Flex solves problems like this by using
the concept of \textit{start conditions}.\cite[10]{flex} A start
condition causes the lexer to enter a state when it matches a
particular rule and only exit the state when an end rule is matched.
Using an exclusive start condition, the lexer will not attempt to
match any rules except for those intended to be matched inside the
start condition---for example, control characters in a JSON
string.\footnote{See json\_xml.l, lines 23-84.}\par
Flex is more powerful than the SLLGEN equivalent, though this power
comes at the expense of complexity.  Flex can ultimately lex more
complexly and with more options than SLLGEN.  The trade-off is worth
it in the case of parsing JSON, since JSON strings cannot be parsed
correctly with SLLGEN.
\subsection{SLLGEN vs. Bison: Parsing}
Bison and the SLLGEN parser work in fundamentally different ways.
SLLGEN generates a parser from an \textit{LL(1)} grammar, specified
using a scheme macro that approximates Backus-Naur form (BNF).  In
contrast, Bison generates a parser using \textit{LR} grammars,
specified in a format closer to written BNF.\par
SLLGEN generates a top-down LL parser, while Bison generates a
shift-and-reduce LR parser.  This leads to significant differences
when specifying grammars for the parser generators.  Since the SLLGEN
parser is LL, it cannot utilize left-recursion to define rules such as
those for JSON objects\cite[pg.67]{compilers}.  In contrast, Bison
prefers left-recursion, since it requires less stack space to
parse.\cite[3.3.3]{bison} The rule for parsing JSON objects
in SLLGEN is generated using the SLLGEN special command for a
seperated list:
\begin{lstlisting}
  (json ("{" (separated-list jstring ":" json ",") "}") jobj)
\end{lstlisting}
In contrast, the rule in Bison is specified using left recursion:
\begin{lstlisting}
  jobj:    jobjpair
        |  jobj LS jobjpair
        ;
\end{lstlisting}
The code necessary to generate the Bison scanner ends up being longer
than SLLGEN's, but the Bison grammar is far more clear.  It more
closely follows BNF, and it is clear from simply looking at the
grammar how tokens will be shifted and reduced.  In contrast, the
SLLGEN grammar is frustratingly abstracted, it is difficult to tell in
the above example how exactly separated-list works, and macros like
this hide some of the actually grammar specification.  In addition,
Bison allows for code to be executed while the parse-tree is being
generated.  This leads to a far more powerful possibilities than simply
generating a parse tree.
\subsection{JSON to XML}
The Bison parser can have a \textit{semantic action} for every
syntactic rule.\cite[3.4.6]{bison}  This is what allows the
parser to do more than run and say if the input is valid or not--it
allows you to construct the parse tree for use in a compiler or
interpreter, or in the case of my JSON to XML translator, simply
translate as the parse tree is constructed.\footnote{This merging of
  the JSON to XML interpreter and the parser is not typical in Bison.
  A parse tree could have been constructed and processed later, but
  this was unnecessary since most of JSON can translate directly to
  XML.} To do this, the JSON to XML parser includes \textit{midrule
  actions}\cite[3.4.8]{bison} as well as end of rule
actions.  For example, these midrule and normal actions print the XML
tags for every item in a list:
\pagebreak
\begin{lstlisting}
  jlist:
           json
         |
           jlist
           { printf("</%s>", list_names[list_depth]); }
           LS
           { printf("<%s>", list_names[list_depth]); }
           json
           { printf("</%s>", list_names[list_depth]); }
         ;
\end{lstlisting}
Writing the JSON to XML interpreter in C is more time consuming than
writing it in Scheme, though it is made easier by combining it with
the Bison generated parser.  Even with this merge, it is still far
faster to write the interpreter in a language that so naturally
handles recursion and has automatic memory management.
\section{Conclusion}
The languages and tools we use to write programs have a direct impact
on our efficacy as programmers, and on the possible properties of our
programs.  This is even more true when writing metaprograms.\par
The complexity of metaprograms makes having tools that are easy to
understand and which promote correct code essential.  After
implementing a JSON to XML interpreter in both Scheme using SLLGEN and
in C using Flex+Bison, it is clear that Flex+Bison are the better
tools for larger projects.  Flex simply allows the programmer to do
more than SLLGEN's lexer.  Bison is far more explicit and powerful
than SLLGEN's parser generator.\par
Despite these advantages for larger projects, the implementation
language of our metaprogramming contestants comes into question for a
small project such as this.  Even while being comfortable in both C
and Scheme, writing an interpreter in C simply takes more time, and
can produce far more errors.  When a programmer wants to spend a
couple hours writing a simple interpreter, SLLGEN is a better fit,
simply because it comes with Scheme.\par
Writing my JSON to XML interpreter using Flex+Bison was fun, but it
also clearly highlights the difference that language and tool design
can make to end-user programmers.
\section{Tests}
Testing for the program was composed of two main parts: testing
correctness of JSON object to XML tag translation, and testing memory
management.  The program correctly translates JSON to XML, with a
couple of exceptions that I did not want to implement:
\begin{itemize}
\item There is no Unicode support.
\item JSON object names are not camel cased when translated to XML tags.
\item If the JSON data type is not enclosed in an object it is not
  converted to accurate XML.  There seems to be no real standard for
  converting JSON not enclosed in objects to XML.
\end{itemize}
Tests can be found in the tests directory.\par
To test, I ran a number of increasingly complex working and broken
JSON examples.  The parser and lexer successfully catch when JSON is
incorrectly formatted. To test memory, I used valgrind to check for
leaks under circumstances when the program terminates.  The parser
generated by Bison does not garbage collect memory required by the
parse stack or a couple other internal functions on exit, so if 3
blocks are not freed on exit this is acceptable.  Any more, and there
is a problem with memory management.  The program does not leak
additional memory under any circumstances.\par
Some basic tests are included in the test directory, along with a
script to easily run more tests.  The process for creating and running
tests is explained in the README.

\pagebreak
\bibliographystyle{plain}
\bibliography{jxmlbib}
\pagebreak
\appendix
\section{JSON String Specification}
\includegraphics[width=\textwidth]{jsonspec}
\end{document}